Les systèmes électroniques numériques sont constitués de bascules (permettant de
sauvegarder des états logiques) et de portes logiques (et, ou, non, ou exclusif par
exemple) interconnectés entre elles. Ces portes sont elles même constituées de
transistors qui sont donc les briques élémentaires de l'électronique numérique. En
effet, ils peuvent être utilisés comme de minuscules interrupteurs pilotés. Il est
possible de graver un enchevêtrement complexe de transistors sur une seule plaque de
silicium, on parle de circuit intégré.  Concevoir un circuit intégré consiste alors à
choisir des portes logiques et des bascules et à définir des connexions entre elles
afin d'obtenir le comportement désiré.  Les portes logiques sont aussi décrites par
des réseaux de transistors.  Au final, un circuit intégré est construit en gravant un
complexe réseau de transistors sur une seule plaque de silicium.

Historiquement ce réseau complexe était dessiné à la main sur des feuilles spéciales
(feuilles de mylar) ce qui limitait fortement la complexité des circuits intégrés. De
plus, il n'était pas possible de tester la conception sans la graver. La montée en
puissance des ordinateurs a alors permis la mise en place de langages de description
de matériel (appelé HDL pour \textit{hardware description language}). Les HDL
permettent de modéliser le comportement des circuits logiques et également de décrire
les structures de ces circuits logiques, on peut alors simuler le comportement de la
structure décrite et vérifier qu'elle se comporte de manière conforme au modèle. Il
est souvent possible de déduire de manière automatique la description structurelle
d'un circuit à l'aide de sa description comportementale. L'apparition de ses langages
a alors révolutionné le monde de la conception des circuits intégrés. Les concepteurs
décrivent alors leurs circuits grâce aux HDL, les simulent de manière virtuelle et
génèrent automatiquement à l'aide de synthétiseurs les dessins physiques qui
représentent le circuit intégré, ces dessins appelés netlists sont ensuite envoyés à
une fonderie qui s'occupe alors de la fabrication. Une fonderie est une entreprise
spécialisée dans la fabrication de circuits. Ces nouvelles techniques de productions
permettent alors de produire des circuits intégrés spécifiques à des attentes
particulières plutôt que d'utiliser des microcontrôleurs et de les programmer afin
d'obtenir le comportement attendu, il s'agit d'ASIC pour \textit{Application Specific
Integrated Circuit}. Cependant la conception et surtout la fabrication d'ASIC ne
peuvent être rentabilisées que pour de gros volumes de production, de plus une erreur
de conception sur un ASIC peut coûter très cher. Mais l'apparition des FPGA (pour
\textit{Field Programmable Gate Array}), qui sont des circuits intégrés logiques
reprogrammables après leur conception, marque encore une fois une révolution.  Ils
permettent de tester de manière plus poussée la conception d'ASIC en créant des
prototypes et peuvent même être vendus une fois programmés à la place d'ASIC pour des
petits volumes de production. Les FPGA de nos jours sont assez performant pour
contenir l'intégralité des composants nécessaires au fonctionnement d'un système
complet (microprocesseurs, mémoire, GPIO par exemple), on parle alors de SoC. Les SoC
(\textit{Systen On Chip}) désignent un système complet qui est contenu sur une seul
puce que ce soit un FPGA ou un ASIC.

Dans notre projet tutoré, nous allons nous intéresser plus particulièrement au
MilkyMist SoC, qui est un SoC open source, c'est à dire que le code HDL qui le décrit
est libre de droit.  Ce SoC a été conçu pour être programmé sur un FPGA spécifique le
Virtex-4 XC4VLX25.  Notre but est d'adapter le code de ce SoC pour le rendre
compatible avec le Xiling Spartan 6 qui est utilisé à l'INSA de Toulouse.  Nous
allons aussi ajouter un module à MilkyMist permettant d'exploiter la propriété de
certains FPGA qui est de pouvoir être reprogrammé à la volée. C'est à dire que nous
allons permettre au système MilkyMist de pouvoir ajouter à chaud (sans redémarrer le
système) des périphériques (à partir de leurs descriptions structurelles) sur le FPGA
et de les connecter au SoC. 

Ce rapport ayant pour but de clarifier le travail de notre projet tutoré, nous allons
dans un premier temps détailler la conception des circuits intégrés.  Pour cela, nous parlerons
dans un premier temps des supports pouvant recevoir des circuits intégrés (FPGA et ASIC),
nous parlerons ensuite de la conception du matériel à l'aide des langages de
description de matériel tel que VHDL, Verilog et Migel et nous ferons un état sur
l'avancement de la communauté Open Source dans ce domaine.  Dans la deuxième partie,
nous parlerons ensuite du projet MilkyMist et de l'intention de notre contribution
sur ce projet.
