Aujourd'hui, les appareils électroniques que nous connaissons ont tous une ou plusieurs fonctionnalités utiles dans telle ou telle situation. Les appareils de téléphonie, de  réseaux, de télécoms, de traitement d'images, de traitement du son possèdent tous des cartes différentes, chacune d'entre elles étant optimisée pour sa fonction. Nombre d'entre eux ne se limitent d'ailleurs pas à une seule carte mais en nécessite plusieurs.\\

Imaginons maintenant qu'un appareil puisse s'adapter à un besoin à court terme. Imaginez un système qui n'a pas de vocation particulière ou plutôt qui en a une infinité et que vous pouvez choisir celle que vous voulez à un instant donné. Voilà pourquoi il devient intéressant d'étudier les systèmes adaptatifs qui vont s'imposer dans notre vie de tous les jours avec le développement et la  diffusion des puces électroniques reconfigurables et reconfigurables à chaud appelées \fpgas{} pour \english{Field-Programmable Gate Array}.\\

Ce projet est une tentative d'exploitation de cette nouvelle technologie portant le nom de \english{partial reconfiguring} qui permet à une puce \fpga{} de se reconfigurer à chaud, c'est-à-dire pendant qu'une partie du \fpga{} est encore active dans une zone particulière de la puce. Nous devions, après avoir étudier le sujet des FPGA, des \english{Systems-On-Chip} et du \english{partial reconfiguring},  adapter le \english{System-On-Chip} open source \textit{Milkymist}, à la carte \nexys{}.  Le but était ensuite d’y intégrer un module de \english{partial reconfiguring}. Cependant certaines difficultés nous ont amenés à concevoir notre propre \english{System-On-Chip} pour la \nexys{} et à étudier la possibilité de faire communiquer ce système avec une autre carte équipé d’un FPGA Virtex5 sur lequel serait mis en place le module de \english{partial reconfiguring}.\\

Ce rapport a pour but d’expliquer le projet et les technologies utilisées. Pour cela, nous relatons notre stratégie d’organisation dans une première partie. Nous détaillons dans une seconde partie le fonctionnement des \fpgas{} et des \english{Systems-On-Chip}. La troisième partie traite de notre tentative de portage du système \textit{Milkymist} et des difficultés rencontrées. La quatrième partie relate la conception d'un nouveau \english{System-On-Chip} développé par l'équipe. Enfin, La cinquième partie fait un bilan du projet et des connaissances acquises.
%Nous verrons dans un première partie comment l'équipe a été organisé, puis nous présenterons le contexte du projet. Premièrement, nous parlerons de la tentative de portage de MilkyMist et ensuite du nouveau système créé à partir de zéro. Enfin, nous ferons le bilan de l'expérience que nous avons acquise.


