
Les \fpgas{} (\english{Field-Programmable Gate Array}) sont des puces qui contiennent un réseau reconfigurable de logique. Les puces peuvent être configurées pour réaliser n'importe quelle fonction logique complexe comme un processeur par exemple. Il existe plusieurs constructeurs de \fpgas{}, les deux plus connus sont \brand{Xilinx} et \brand{Altera}. La configuration d'un \fpga{} est obtenue après ce qu'on appelle la synthèse d'un code écrit en langage HDL (\english{Hardware Description Language}) et le routage des liens entre les différentes logiques. Le fichier généré s'appelle un \english{bitstream}. Ces langages sont bien différents des langages de programmation informatique puisqu'ils permettent de décrire un fonctionnement et ne correspondent pas à un algorithme. Ce type de langage permet de faire de la conception \english{hardware} à moindre coût puisqu'on peut réutiliser le même \fpga{} pour plusieurs conceptions. Le code HDL pourra ensuite être utilisé pour graver directement des puces ASIC (\english{Application-Specific Integrated Circuit}) et obtenir un produit fini.\\

Les \fpgas{} sont constitués d'un nombre très importants de blocs logiques organisés en matrices (d'où le terme \english{Array} dans \fpga{}). Ces blocs logiques élémentaires (AND, OR, XOR, NOT) peuvent être configurés pour effectuer d'autres fonctions logiques bien plus complexes. Il est aussi possible d'utiliser des bascules ou d'autres éléments permettant de mémoriser de l'information. Les nouveaux \fpgas{} contiennent des blocs de RAM de plusieurs kilo octets. Grâce au code HDL, un réseau est tissé et relie toute cette logique pour former un système complexe correspondant à la fonction attendue.\\

De nos jours, les transistors dans les \fpgas{} se comptent en millions. Les puces \fpga{} ne sont en général qu'une partie d'un système plus général. Elles sont implantés sur des circuits imprimés et sont reliées à d'autres périphériques externes comme de la mémoire RAM ou flash, un GPIO (\english{General Purpose Input/Output}), une carte audio, une carte Ethernet. Dans ce projet, nous utilisons une carte de développement appelée \nexys{} développée par l'entreprise \brand{Digilent} qui comprend entre autre une puce \fpga{} Xilinx Spartan 6, un bloc de mémoire RAM et Flash, des LEDS, des boutons, un port Ethernet, un port VGA pour l'affichage sur un moniteur et un port USB.