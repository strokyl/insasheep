Ce projet a pour origine le projet de fin d'étude de Sébastien Bourdeauducq \cite{BOURDEAUDUCQ} sur le design d'un System-on-Chip au Royal Institute of Technology de Stockholm en 2010. Plus précisément, il s'agissait de réaliser un SoC rapide et économe en ressources, basé sur FPGA et avec pour but principal de supporter une application de rendu d'effets video en temps réel. Ces effets peuvent être aussi bien réalisés avec un ordinateur normal. Cependant cette approche a certains inconvénients, un système embarqué a l'avantage d'être moins encombrant et plus facile à mettre en place. Le temps de démarrage et de configuration est réduit à  quelques secondes contairament à un ordinateur normal qui doit d'abord lancer un système d'exploitation lourd tel que Windows et ensuite lancer un logiciel spécialisé permettant de créer des effets vidéo. De plus, on peut avoir des interfaces spécialisées qu'un PC normal n'a pas, à moins de les rajouter, ce qui peut se révéler assez cher.


Le SoC \textit{Milkymist} est implémenté sur un FPGA et utilise le coeur LatticeMico32 (LM32)\cite{LATTICE}. C'est un CPU 32-bit \textit{big endian} RISC \textit{open-source}. Le microprocesseur LM32 est assisté dans son travail par une unité de mappage de texture et par un coprocesseur programmable à virgule flottante qui est utilisé par le logiciel de synthèse video.