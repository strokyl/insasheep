\vspace{15px}

Ce projet a pour origine la thèse de Sébastien Bourdeauducq sur le design d'un System-on-Chip au Royal Institute of Technology de Stockholm en 2010. Plus précisément, il s'agissait de réaliser un SoC rapide et économe en ressources basé sur FPGA et avec pour principal but de supporter une application de rendu d'effets video en temps réel. Ces effets peuvent être aussi bien réalisé avec un ordinateur normal. Cependant cette approche a certains inconvénients, un système embarqué a l'avantage d'être moins encombrant et plus facile à mettre en place. Le temps de démarrage et de configuration est réduit à  quelques secondes contairament à un ordianteur normal qui doit d'abord lancer un système d'exploitation lourd tel que Windows et ensuite lancer un logiciel spécialisé permettant de créer des effets vidéo. De plus, on peut avoir des interfaces spécilisées qu'un PC normal n'a pas à moins de les rajouter, ce qui peut se révéler assez cher.
Le SoC Milkymist est implémenté sur un FPGA, il utilise le coeur LatticeMico32 (LM32). C'est un CPU 32-bit big endian RISC Opensources. Le microprocesseur ML32 est assisté dans son travail par une unité de mappage de texture et par un coprocesseur programmable à virgule flottante qui est utilisé par le logiciel de synthèse video.
\medskip