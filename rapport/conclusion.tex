Comme nous l'avons vu au cours de ce rapport les FPGA permettent bien plus de chose que
de juste prototyper des ASIC. Leur puissance et leur nombre de portes grandissant
leur permet maintenant d'acceuillir des systèmes complets tel que MilkyMist. De plus
leurs facultés à pouvoir se reconfigurer à chaud ouvrent de nombreuses portes qui
pourraient bien révolutionner l'informatique. Nous comptons alors ajouter un module
au projet Open Source MilkyMist permettant d'utiliser cette caracteristique. Nous
avons déjà réussi à porter le projet sur la carte de développement Nexys 3 utilisant
le fpga Spartan 6 de Xiling qui nous est fourni par l'INSA. Le module que nous allons
concevoir permettrat alors de rajouter à chaud des périphériques au System On Chip
MilkyMist selon les besoins du moment.
