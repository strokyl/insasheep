Comme nous l'avons vu au cours de ce rapport les FPGA permettent bien plus de choses que
de prototyper des ASIC. Leur puissance et leur nombre de portes grandissant
leur permettent maintenant d'acceuillir des systèmes complets tels que \textit{Milkymist}. De plus,
leurs facultés à pouvoir se reconfigurer à chaud ouvrent de nombreuses portes qui
pourraient bien révolutionner l'informatique. Nous comptons alors ajouter un module
au projet \textit{open-source Milkymist} permettant d'utiliser cette caracteristique. Nous
avons déjà réussi à porter le projet sur la carte de développement Nexys3 utilisant
le FPGA Spartan6 de Xilinx fourni par l'INSA. Le module que nous allons
concevoir permettra alors de rajouter à chaud des périphériques au \textit{System-on-Chip
Milkymist} selon les besoins du moment.
