\bibitem{RUBOW}
RUBOW, Erik. \textit{Open Source Hardware}. \textbf{[en ligne]}. Etats-Unis d'Amérique : University of California, San Diego, Cours, 2008, 5p. Disponible sur <http://cseweb.ucsd.edu/classes/fa08/cse237a/topicresearch/erubow\_tr\_report.pdf> (Consulté le 24 janvier 2013).

\bibitem{ACOSTA}
ACOSTA, Roberto. \textit{Open Source Hardware}. \textbf{[en ligne]}. Etats-Unis d'Amérique : Massachusetts Institute of Technology, 2009, 83p. Disponible sur <http://18.7.29.232/bitstream/handle/1721.1/55201/609411873.pdf> (Consulté le 24 janvier 2013).

\bibitem{BAXTER_BENNETT}
BAXTER, Julius, BENNETT, Jeremy et opencores.org. \textit{Practical System-on-Chip}. \textbf{In} : \textit{Open Source Hardware User Group} \textbf{[en ligne]}. (Modifié le 29 mars 2012). Disponible sur <http://oshug.org/presentations/openrisc-oshug17-final.pdf> (Consulté le 24 janvier 2013).

\bibitem{GAISLER}
GAISLER, Jiri. \textit{Industrial use of open-source IP cores}. \textit{45th IFIP WG 10.4 - Workshop on "Open Source and Dependability"}. Moorea polynésie française, 5-9 mars 2004. Disponible sur <http://webhost.laas.fr/TSF/IFIPWG/Workshops\&Meetings/45/06-Gaisler.pdf> (Consulté le 24 janvier 2013).

\bibitem{INMAN_FRITSKY}
INMAN, Kurt and FRITSKY, Lauren. \textit{What Is an IP Core?}. \textbf{In} : \textit{wiseGEEK} \textbf{[en ligne]}. Disponible sur <http://www.wisegeek.com/what-is-an-ip-core.htm> (Consulté le 24 janvier 2013).

\bibitem{RUBOW}
SALEM, Mohamed A. and KHATIB, Jamil I. An introduction to open-source hardware development. EETimes. Modifié le 07 janvier 2004. Disponible sur <http://eetimes.com/electronics-news/4155052/An-introduction-to-open-source-hardware-development>.

\bibitem{OSHW}
OSHW. \textit{Definition of Free Cultural Works} \textbf{[en ligne]}. (modifié le 9 janvier 2013). Disponible sur : http://freedomdefined.org/OSHW. (Consulté le 24 janvier 2013).

\bibitem{WITZ_CHAMBON}
WITZ, Jean-François, CHAMBON, Olivier. \textit{Matériel OpenSource (OpenHardware) et la recherche}. \textbf{In} : \textit{PLUME} \textbf{[en ligne]}. (Modifié le 29 août 2011). Disponible sur <https://www.projet-plume.org/ressource/openhardware> (Consulté le 24 janvier 2013).

\bibitem{TORRONE_FRIED}
TORRONE, Phillip et FRIED, Limor. \textit{Million dollar baby}. \textit{Foo Camp East 2010} (http://foocamp10.wiki.oreilly.com/wiki/index.php/Main\_Page). Sebastopol, Californie, 25-27 juin 2010. Disponible sur <http://www.adafruit.com/pt/fooeastignite2010.pdf> (Consulté le 27 janvier 2013).

\bibitem{LIST_OSHW_WIKI}
List of open-source hardware projects. \textit{Wikipedia} \textbf{[en ligne]}. (modifié le 17 janver 2013). Disponible sur : http://en.wikipedia.org/w/index.php?title=List\_of\_open-source\_hardware\_projects\&oldid=533515806. (Consulté le 27 janvier 2013).

%DUZAN Luc


\bibitem{ASIC_WIKI}
Application-specific integrated circuit. \textit{Wikipedia} \textbf{[en ligne]}. (Modifié le 08 janver 2013). Disponible sur : http://en.wikipedia.org/wiki/Application-specific\_integrated\_circuit. (Consulté le 20 janvier 2013).

\bibitem{PLD_WIKI}
Programmable logic device. \textit{Wikipedia} \textbf{[en ligne]}. (Modifié le 11 janver 2013). Disponible sur : http://en.wikipedia.org/wiki/Programmable\_logic\_device. (Consulté le 20 janvier 2013).

\bibitem{POITI}
POITI, Alexis. \textit{Cours en ligne de {TELECOM} Paristech sur les {HDL}}. \textbf{[en ligne]}. France : Paristech, Cours en ligne, 2005. Disponible sur <http://comelec.enst.fr/hdl/> (Consulté le 24 janvier 2013).

\bibitem{FOUND_MODEL_WIKI}
Foundry model. \textit{Wikipedia} \textbf{[en ligne]}. (Modifié le 09 janver 2013). Disponible sur : http://en.wikipedia.org/wiki/Foundry\_model. (Consulté le 20 janvier 2013).

\bibitem{NETLIST_WIKI}
Netlist. \textit{Wikipedia} \textbf{[en ligne]}. (Modifié le 09 janver 2013). Disponible sur : http://en.wikipedia.org/wiki/Netlist. (Consulté le 20 janvier 2013).

%fin luc 

\bibitem{RTL_WIKI}
Register-transfer level. \textit{Wikipedia} \textbf{[en ligne]}. (Modifié le 01 janver 2013). Disponible sur : http://en.wikipedia.org/w/index.php?title=Register-transfer\_level\&oldid=530759637. (Consulté le 24 janvier 2013).

\bibitem{THOMAS_MOORBY}
THOMAS, Donald E. et MOORBY, Philip R. \texit{The Verilog® Hardware Description Language.} \textbf{[en ligne]}. Springer. Etats-Unis d'Amérique : 2002, 408p. Format google book. Disponible sur <http://books.google.fr/books?id=DxQGrz7q-SwC\&printsec=frontcover\&hl=fr#v=onepage\&q\&f=false> (Consulté le 24 janvier 2013).

\bibitem{MILKY_WIKI}
Milkymist. \textit{Wikipedia} \textbf{[en ligne]}. (Modifié le 13 janvier 2013). Disponible sur : http://en.wikipedia.org/w/index.php?title=Milkymist\&oldid=532830955 (Consulté le 24 janvier 2013).

\bibitem{KASHAP}
KASHAP, Satish. Lec 08 Hardware Description Language (HDL). [18 mars 2012] [enregistrement vidéo] In : Youtube [1h 07' 54"]. Disponible sur : http://www.youtube.com/watch?v=rdAPXzxeaxs. (Consulté le 20 janvier 2013).

\bibitem{GIOMI}
GIOMI, Jean-Charles et TARROUX, Gerard. \textit{Integrated circuit fabrication using state machine extraction from ...}. Brevet 5537580. 21 décembre 1994.

\bibitem{HDL_WIKI}
Hardware description language. \textit{Wikipedia} \textbf{[en ligne]}. (Modifié le 14 janvier 2013). Disponible sur : http://en.wikipedia.org/w/index.php?title=Hardware\_description\_language\&oldid=532956772 (Consulté le 24 janvier 2013).

\bibitem{SMITH}
SMITH, Douglas J. VHDL \& Verilog Compared \& Contrasted Plus Modeled Example Written in VHDL, Verilog and C. In Rajesh Bawankule's Verilo Center. \textbf{[en ligne]}. (Denière mise à jour du site : 12 février 2003) Disponible sur http://www.angelfire.com/in/rajesh52/verilogvhdl.html (Consulté le 07 janvier 2013).

\bibitem{CHAKRABARTI}
CHAKRABARTI, Indrajit. \textit{VERILOG Hardware Description Language} \textbf{[en ligne]}. Inde : IIT Kharagpur, Cours, 2007, 24p. Dispobible sur <http://www.smdp.iitkgp.ernet.in/PDF\%5CVLSI_DSP\%5CIC-Verilog.pdf> (Consulté le 05 janvier 2013).

\bibitem{VERILOG_WIKI}
Verilog. \textit{Wikipedia} \textbf{[en ligne]}. (Modifié le 7 septembre 2012). Disponible sur : http://fr.wikipedia.org/w/index.php?title=Verilog\&oldid=82794321. (Consulté le 20 janvier 2013).

\bibitem{DON}
DON, Thomas. \textit{The Verilog Hardware Description Language}. \textbf{[en ligne]}. Etats-Unis d'Amérique : University of Iowa, Cours, 1998, 11p. Disponible sur <http://www.engineering.uiowa.edu/~hpca/LectureNotes/VerilogTutorialSpring11.pdf> (Consulté le 05 janvier 2013).

\bibitem{VHDL_WIKI}
VHDL. \textit{Wikipedia} \textbf{[en ligne]}. (Modifié le 4 décembre 2012). Disponible sur : http://fr.wikipedia.org/w/index.php?title=VHDL\&oldid=86139857. (Consulté le 20 janvier 2013).

\bibitem{IEEE_VHDL_87}
1076-1987 - IEEE Standard VHDL Language Reference. \textit{IEEE Standard Association} \textbf{[en ligne]}. Disponible sur http://standards.ieee.org/findstds/standard/1076-1987.html (Consulté le 20 janvier 2013).

\bibitem{IEEE_VHDL_93}
1076-1993 - IEEE Standard VHDL Language Reference. \textit{IEEE Standard Association} \textbf{[en ligne]}. Disponible sur http://standards.ieee.org/findstds/standard/1076-1993.html (Consulté le 20 janvier 2013).

\bibitem{IEEE_VHDL}
P1076 - VHDL Analysis and Standardization Group. \textit{IEEE Standard Association} \textbf{[en ligne]}. Disponible sur http://standards.ieee.org/develop/wg/P1076.html (Consulté le 20 janvier 2013).

\bibitem{IEEE_VERILOG_1995}
1364-1995 - IEEE Standard Hardware Description Language Based on the Verilog(R) Hardware Description Language \textit{IEEE Standard Association} \textbf{[en ligne]}. Disponible sur http://standards.ieee.org/findstds/standard/1364-1995.html (Consulté le 20 janvier 2013).

\bibitem{IEEE_VERILOG_2001}	
1364-2001 - IEEE Standard Verilog Hardware Description Language. \textit{IEEE Standard Association} \textbf{[en ligne]}. Disponible sur http://standards.ieee.org/findstds/standard/1364-2001.html (Consulté le 20 janvier 2013).

\bibitem{IEEE_VERILOG_2005}
1364-2005 - IEEE Standard for Verilog Hardware Description Language. \textit{IEEE Standard Association} \textbf{[en ligne]}. Disponible sur http://standards.ieee.org/findstds/standard/1364-2005.html (Consulté le 20 janvier 2013).

%clem
\bibitem{RECOMPUT_WIKI}
Reconfigurable Computing Wikipedia. \textit{Wikipedia} \textbf{[en ligne]}. (modifié le 01 janvier 2013). Disponible sur : http://en.wikipedia.org/wiki/Reconfigurable\_computing. (Consulté le 16 janvier 2013).

\bibitem{MILKY_SITE}
Milkymist official website. \textit{Milkymist} \textbf{[en ligne]}. Disponible sur : http://milkymist.org/3/. (Consulté le 20 janvier 2013).


%matt
\bibitem{SALEH_MIRABBASI}
SALEH, Resve; MIRABBASI, Shahriar; LEMIEUX, Lemieux; GRECU, Cristian. System-on-Chip: Reuse and Integration \textbf{[en ligne]. In :} \textit{Proceedings of the IEEE}, vol. 94, No. 6, June 2006. Disponible sur : http://eecs.wsu.edu/~pande/Journal\_Papers/Paper\_IEEE\_Proceedings.pdf (Consulté le 05/01/2013)

\bibitem{WINZKER}
M. WINZKER. \textit{Advenced FPGA Design}. \textbf{[en ligne]} Buenos Aires : Universidad Tecnológica Nacional. Cours, 2010, 15 slides. Disponible sur : http://www.electron.frba.utn.edu.ar/dplab/CursoLP/Lecture\_FPGA\_5.pdf (Consulté le 05/01/2013)

\bibitem{BOURDEAUDUCQ}
BOURDEAUDUCQ, Sébastien. \textit{A performance-driven SoC architecture for video synthesis}. \textbf{[en ligne]} Master of Science Thesis in System-on-Chip Design, Stockholm, Royal Institute of Technology, 2010. 109. Disponible sur : http://milkymist.org/3/thesis/thesis.pdf (Consulté le 01/01/2013)