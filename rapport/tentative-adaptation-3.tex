\subsection{Quelques difficultés}

Le projet MilkyMist s'est révélé être trop important et complexe pour être pris en main de manière rapide et efficace. La partie hardware contenait beaucoup de modules inutiles et le design du système en lui-même n'était pas très adapté à notre utilisation. Il était difficile d'éviter les effets de bords lorsqu'on supprimait un module. La simulation du système était une tâche dont l'ampleur avait été sous-estimée.\\
Nous avons également été confronté à des problèmes au niveau du système qui ne démarre pas sur la \nexys{}. Plusieurs pistes ont été explorées mais aucune n'a porté ses fruits. Le travail a été d'autant plus compliqué que nous ne possédions pas un système complet et fonctionnel (nous n'avions pas de RAM). Une dernière piste possible, non explorée à ce jour par manque de temps, est le \english{Device Tree Source} ou \textit{DTS}. Ce fichier permet au noyau de connaître ses périphériques mais puisque nous ne travaillons pas sur la même carte que celle de MilkyMist, les périphériques ne sont pas les mêmes et les adresses où ils se trouvent sont différentes. Ce problème est d'autant plus ennuyant que sur cible simulée, le noyau et son système de fichier fonctionnent parfaitement.\\
Le temps d'installation de tous les outils sur les machines est très long à cause notamment des bugs de version. La compilation du compilateur pour LM32, l'installation de ISE, les problèmes de bibliothèques manquantes ont aussi été problématiques. Les différentes versions de linux utilisées ont posé des problèmes de compatibilités qui ont obligé certains membres de l'équipe à créer une \english{sandbox} afin d'exécuter des programmes compilés sur d'autres machines.

\subsection{Réorientation vers une nouvelle solution \english{from scratch}}

Le système \textit{Milkymist} est un système complexe et il a paru plus facile d'en recréer un avec les modules nécessaires mais en partant de zéro. Tout refaire depuis le début nous a donné l'avantage d'avoir une meilleure maîtrise du code. Il en a été de même pour le software puisqu'il a été décidé de créer notre propre kernel.