Bien que nous n'ayons pu effectuer qu'une petite partie de ce que nous avions prévu, ce projet nous a énormément appris sur le plan technique, théorique et organisationnel.\\
La \english{partial reconfiguration} est un sujet passionnant avec un potentiel d'application énorme. Il permet de réduire la frontière entre logiciel et matériel. L'arrivée des \fpgas{} avait déjà révolutionné l'électronique numérique en permettant la reprogrammation d'un circuit logique. Avec la \english{partial reconfiguration}, le matériel n'est plus figé après le démarrage du système d'exploitation, il peut évoluer tout comme le contenu de la mémoire et change la manière de concevoir les systèmes embarqués.\\
Nous avons tenté d'utiliser cette technologie en reprenant le projet \textit{Milkymist}. Malgré nos difficultés à adapter le SoC, nous avons su rebondir en créant notre propre \english{System-On-Chip}. Le codage d'un mini-noyau nous a apporté beaucoup de connaissances sur le fonctionnement des systèmes d'exploitation.\\
Bien que nous n'ayons pas pu le finaliser, le projet est suffisamment mature pour servir de support dans les années à venir. Le domaine de la \english{partial reconfiguration} est une terre vierge qui pourrait faire naître bon nombre de travaux de recherche.

