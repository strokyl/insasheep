\vspace{15px}
Dans cette partie nous allons parler des matériels open-source, c'est à dire que les matériels, ou plutôt leurs conceptions, sont crées de manière communautaire ou par des sociétés privée. Ils ne sont pas soumis à des brevets ultra-restricitfs mais qui sont le plus souvent sous licence GPL (General Public License) ou assimilé plus souples.
\medskip
Le materiel open-source est très peu developpé contrairement à son homologue loigiciel. Il n'y a que peu d'acteurs sur cette scène alors qu'il y en a des dizaines pour l'open-source logiciel. Cependant, ce pan de l'open-source tend de plus en plus à se developper ces derniers temps. Pour preuve, en 2010 a été publié le \textbf{Open Source Hardware} (\textbf{OSHW}). Il s'agit d'une définition de ce qu'est un matériel open-source et des idéaux de ce dernier.
\medskip
La plus grosse communauter à l'heure actuelle est OpenCores (http://opencores.org). Fondée en 1999 par Damjan Lampret, OpenCores (OC) a commendé a developpé \textit{OpenRISC 1000}. Le développent s'est poursuivi jusqu'en 2007. Durant cette période OC a été supporté par \textit{Flextronics}. A partir de 2007, \textit{OpenRISC 1000} est passé dans une phase de déploiement commerciale grâce à \textit{ORSoC AB}. En mars 2012, il y avait plus de 147 000 utitilisateurs enregistrés et plus de 900 projets en cours sur cette architecure matériel. \textit{ORPSoC} est l'implémentation de SoC de référence sur cette architecture.
\medskip
